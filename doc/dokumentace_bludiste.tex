
\documentclass[12pt,a4paper]{article}
\usepackage[czech]{babel}
\usepackage[utf8]{inputenc}
\usepackage[T1]{fontenc}
\usepackage{geometry}
\geometry{margin=2.5cm}
\usepackage{hyperref}
\usepackage{titlesec}
\titleformat{\section}{\large\bfseries}{\thesection.}{1em}{}

\begin{document}

\noindent
Souhlasím s vystavením této semestrální práce na stránkách katedry informatiky a výpočetní techniky a jejímu využití pro prezentaci pracoviště.

\vspace{1cm}

\begin{flushleft}
\textbf{Jméno:} Vít Novotný \\
\textbf{Datum:} \today \\
\textbf{E-mail:} vnovotny@students.zcu.cz \\
\textbf{Studijní číslo:} A23B0412P
\end{flushleft}

\vspace{0.5cm}

\section*{Dokumentace k semestrální práci: 3D bludiště v OpenTK}

\section{Popis projektu}
Aplikace implementuje základní 3D bludiště v prostředí OpenGL pomocí knihovny OpenTK, včetně kompletní fyziky pohybu hráče, kolizí, osvětlení a práce s texturami. Cílem bylo vytvořit interaktivní first-person prostředí, které je stabilní, fyzikálně konzistentní a odpovídá specifikaci zadání.

\section{Klíčové části implementace}

\subsection*{Pohyb hráče a fyzika}
Pohyb hráče je založen na jednoduché fyzikální simulaci – akcelerace je odvozena ze součtu sil působících na hráče. Ovládání reaguje na vstupy (WSAD, Shift pro běh, mezerník pro skok) a generuje síly, které určují směr pohybu.

Hráč se pohybuje konstantní rychlostí bez ohledu na kombinaci směrových kláves. Zároveň je implementována korektní simulace skákání s gravitací a zachycením dopadu.

\subsection*{Kolizní systém}
Byl implementován systém kolizí postavený na testování průniku koule s trojúhelníky pomocí metody ``move and slide''. Tento systém zajišťuje plynulý pohyb podél stěn bez zasekávání nebo poskakování a brání opuštění mapy, i v případě, že některé zdi chybí.

Po každém kroku je upravena pozice hráče podle kolizních korekcí a \texttt{Velocity} je znovu vypočtena podle skutečného posunu, čímž se udržuje fyzikální konzistence.

\subsection*{Osvětlení: Svítilna}
Scéna je osvětlena reflektorem ve výšce 2{,}05 m s depresí 2°. Světlo sleduje pozici a směr hráče, připomíná baterku. Úhel světelného kužele je menší než FOV pozorovatele a osvětluje jen přímý výhled.

Implementace využívá OpenGL spotlight s nastavením směru a cutoff úhlu.

\subsection*{Texturování}
Zdi a podlaha jsou texturované pomocí klasického vzorkování. Načítání textur je provedeno s ohledem na HW podporu – ve výchozím nastavení je použito lineární filtrování s mipmapami.

\subsection*{FPS Counter}
Na obrazovce je zobrazeno počítadlo snímků za vteřinu, které počítá, kolik snímků proběhlo za poslední jednu vteřinu. Výstup je nezávislý na snímkové frekvenci a zajišťuje reálný odhad výkonu.

\subsection*{First-person pohled a ovládání}
Kamera sleduje hráče z pohledu první osoby. Myší lze otáčet pohled (vertikálně i horizontálně), přičemž rotace je omezena v rozsahu -90° až +90° ve vertikále. FOV lze dynamicky měnit kolečkem myši mezi 30° a 120°, výchozí hodnota je 90°.

\section{Implementované prvky dle zadání}

\subsection*{Povinná část}
\begin{itemize}
    \item Pravoúhlá síť bludiště
    \item Fyzikálně založený pohyb hráče (rovnoměrná rychlost, vektorový směr)
    \item Kolize se stěnami bez poskakování a průchodů
    \item Zajištění hranic mapy i bez zdi
    \item Plynulé řízení pohledu myší, omezení přetočení
    \item Vše nezávislé na FPS
    \item Osvětlení svítilnou dle specifikace
    \item FPS counter
    \item Měřítko a geometrie dle zadání (1 jednotka = 1 m)
\end{itemize}

\subsection*{Volitelná část}
\begin{itemize}
    \item Texturování zdí a podlahy
    \item Svítilna (projekční světlo)
    \item FPS měření
    \item Pokročilý kolizní systém
\end{itemize}

\section{Poznámky k ovládání}
\begin{itemize}
    \item \texttt{W, A, S, D} – pohyb vpřed, vlevo, vzad, vpravo
    \item \texttt{Shift} – běh
    \item \texttt{Mezerník} – skok
    \item \texttt{Myš} – otáčení pohledu
    \item \texttt{Kolečko} – změna FOV (30°–120°)
\end{itemize}

\section{Závěr}
Implementace naplňuje požadavky zadání a zároveň ukazuje silnou stránku ve fyzikálně korektním chování hráče, precizních kolizích a světelném modelu. Kód je rozdělen do samostatných modulů, což usnadňuje rozšiřování o další funkce (např. minimapa, AI, grafické efekty).

\end{document}
