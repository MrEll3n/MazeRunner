
\documentclass[12pt,a4paper]{article}
\usepackage[czech]{babel}
\usepackage[utf8]{inputenc}
\usepackage[T1]{fontenc}
\usepackage{geometry}
\geometry{margin=2.5cm}
\usepackage{hyperref}
\usepackage{titlesec}
\usepackage{listings}
\usepackage{xcolor}
\lstset{
  basicstyle=\ttfamily\small,
  keywordstyle=\color{blue},
  commentstyle=\color{gray},
  stringstyle=\color{orange},
  breaklines=true,
  frame=single,
  columns=fullflexible,
  keepspaces=true,
  language=C,
  morekeywords={Vector3, return, float, if, else, static, out},
  extendedchars=true,
  inputencoding=utf8,
  literate=%
    {á}{{\'a}}1 {č}{{\v{c}}}1 {ď}{{\v{d}}}1 {é}{{\'e}}1 {ě}{{\v{e}}}1
    {í}{{\'i}}1 {ň}{{\v{n}}}1 {ó}{{\'o}}1 {ř}{{\v{r}}}1 {š}{{\v{s}}}1
    {ť}{{\v{t}}}1 {ú}{{\'u}}1 {ů}{{\r{u}}}1 {ý}{{\'y}}1 {ž}{{\v{z}}}1
    {Á}{{\'A}}1 {Č}{{\v{C}}}1 {Ď}{{\v{D}}}1 {É}{{\'E}}1 {Ě}{{\v{E}}}1
    {Í}{{\'I}}1 {Ň}{{\v{N}}}1 {Ó}{{\'O}}1 {Ř}{{\v{R}}}1 {Š}{{\v{S}}}1
    {Ť}{{\v{T}}}1 {Ú}{{\'U}}1 {Ů}{{\r{U}}}1 {Ý}{{\'Y}}1 {Ž}{{\v{Z}}}1
}
\titleformat{\section}{\large\bfseries}{\thesection.}{1em}{}

\begin{document}

\noindent
Souhlasím s vystavením této semestrální práce na stránkách katedry informatiky a výpočetní techniky a jejímu využití pro prezentaci pracoviště.

\vspace{1cm}

\begin{flushleft}
\textbf{Jméno:} Vít Novotný \\
\textbf{Datum:} \today \\
\textbf{E-mail:} vnovotny@students.zcu.cz \\
\textbf{Studijní číslo:} A23B0412P
\end{flushleft}

\vspace{0.5cm}

\section*{Dokumentace k semestrální práci: 3D bludiště v OpenTK}

\section{Popis projektu}
Aplikace implementuje základní 3D bludiště v prostředí OpenGL pomocí knihovny OpenTK, včetně kompletní fyziky pohybu hráče, kolizí, osvětlení a práce s texturami. Cílem bylo vytvořit interaktivní first-person prostředí, které je stabilní, fyzikálně konzistentní a odpovídá specifikaci zadání.

\section{Klíčové části implementace}

\subsection*{Pohyb hráče a fyzika}
Pohyb hráče je založen na jednoduché fyzikální simulaci – akcelerace je odvozena ze součtu sil působících na hráče. Ovládání reaguje na vstupy (WSAD, Shift pro běh, mezerník pro skok) a generuje síly, které určují směr pohybu.

Hráč se pohybuje konstantní rychlostí bez ohledu na kombinaci směrových kláves. Zároveň je implementována korektní simulace skákání s gravitací a zachycením dopadu.

\subsection*{Kolizní systém}
Byl implementován systém kolizí postavený na testování průniku koule s trojúhelníky pomocí metody ``move and slide''. Tento systém zajišťuje plynulý pohyb podél stěn bez zasekávání nebo poskakování a brání opuštění mapy, i v případě, že některé zdi chybí.

Po každém kroku je upravena pozice hráče podle kolizních korekcí a \texttt{Velocity} je znovu vypočtena podle skutečného posunu, čímž se udržuje fyzikální konzistence.

\newpage
Následující metoda určuje nejbližší bod ke kouli na trojúhelníku ve 3D prostoru. Používá se při testu průniku:

\begin{lstlisting}
private static Vector3 ClosestPointOnTriangle(Vector3 p, Vector3 a, Vector3 b, Vector3 c)
{
    Vector3 ab = b - a;
    Vector3 ac = c - a;
    Vector3 ap = p - a;

    float d1 = Vector3.Dot(ab, ap);
    float d2 = Vector3.Dot(ac, ap);
    if (d1 <= 0 && d2 <= 0) return a;

    Vector3 bp = p - b;
    float d3 = Vector3.Dot(ab, bp);
    float d4 = Vector3.Dot(ac, bp);
    if (d3 >= 0 && d4 <= d3) return b;

    float vc = d1 * d4 - d3 * d2;
    if (vc <= 0 && d1 >= 0 && d3 <= 0)
    {
        float v = d1 / (d1 - d3);
        return a + ab * v;
    }

    Vector3 cp = p - c;
    float d5 = Vector3.Dot(ab, cp);
    float d6 = Vector3.Dot(ac, cp);
    if (d6 >= 0 && d5 <= d6) return c;

    float vb = d5 * d2 - d1 * d6;
    if (vb <= 0 && d2 >= 0 && d6 <= 0)
    {
        float w = d2 / (d2 - d6);
        return a + ac * w;
    }

    float va = d3 * d6 - d5 * d4;
    if (va <= 0 && (d4 - d3) >= 0 && (d5 - d6) >= 0)
    {
        float w = (d4 - d3) / ((d4 - d3) + (d5 - d6));
        return b + (c - b) * w;
    }

    Vector3 n = Vector3.Cross(ab, ac);
    float denom = Vector3.Dot(n, n);
    float u = Vector3.Dot(Vector3.Cross(ap, ac), n) / denom;
    float v2 = Vector3.Dot(Vector3.Cross(ab, ap), n) / denom;

    return a + ab * u + ac * v2;
}
\end{lstlisting}

\newpage
\noindent
Další metoda kontroluje, zda koule zasahuje do trojúhelníku, a v případě kolize vrací vektor vytlačení:

\begin{lstlisting}
public static bool SphereIntersectsTriangle(Vector3 center, float radius, Vector3 a, Vector3 b, Vector3 c, out Vector3 pushOut)
{
    pushOut = Vector3.Zero;

    // Najdi nejbližší bod na trojúhelníku k centru koule
    Vector3 closest = ClosestPointOnTriangle(center, a, b, c);
    Vector3 diff = center - closest;
    float distSq = diff.LengthSquared;

    // Pokud je vzdálenost menší než poloměr, nastala kolize
    if (distSq < radius * radius)
    {
        float dist = MathF.Sqrt(distSq);
        Vector3 normal = dist > 0 ? diff / dist : Vector3.UnitY; // pokud je střed přímo na trojúhelníku
        float penetration = radius - dist;
        pushOut = normal * penetration; // vektor, kterým se koule vytlačí z kolize
        return true;
    }

    return false;
}
\end{lstlisting}

\newpage
\subsection*{Osvětlení: Svítilna}
Scéna je osvětlena reflektorem ve výšce 2{,}05 m s depresí 2°. Světlo sleduje pozici a směr hráče, připomíná baterku. Úhel světelného kužele je menší než FOV pozorovatele a osvětluje jen přímý výhled.

Implementace využívá OpenGL spotlight s nastavením směru a cutoff úhlu.

\subsection*{Texturování}
Zdi a podlaha jsou texturované pomocí klasického vzorkování. Načítání textur je provedeno s ohledem na HW podporu – ve výchozím nastavení je použito lineární filtrování s mipmapami.

\subsection*{FPS Counter}
Na obrazovce je zobrazeno počítadlo snímků za vteřinu, které počítá, kolik snímků proběhlo za poslední jednu vteřinu. Výstup je nezávislý na snímkové frekvenci a zajišťuje reálný odhad výkonu.

\subsection*{First-person pohled a ovládání}
Kamera sleduje hráče z pohledu první osoby. Myší lze otáčet pohled (vertikálně i horizontálně), přičemž rotace je omezena v rozsahu -90° až +90° ve vertikále. FOV lze dynamicky měnit kolečkem myši mezi 30° a 120°, výchozí hodnota je 90°.

\section{Implementované prvky dle zadání}

\subsection*{Povinná část}
\begin{itemize}
    \item Pravoúhlá síť bludiště
    \item Fyzikálně založený pohyb hráče (rovnoměrná rychlost, vektorový směr)
    \item Kolize se stěnami bez poskakování a průchodů
    \item Zajištění hranic mapy i bez zdi
    \item Plynulé řízení pohledu myší, omezení přetočení
    \item Vše nezávislé na FPS
    \item Osvětlení svítilnou dle specifikace
    \item FPS counter
    \item Měřítko a geometrie dle zadání (1 jednotka = 1 m)
\end{itemize}

\subsection*{Volitelná část}
\begin{itemize}
    \item Texturování zdí a podlahy
    \item Svítilna (projekční světlo)
    \item FPS měření
    \item Pokročilý kolizní systém
\end{itemize}

\section{Poznámky k ovládání}
\begin{itemize}
    \item \texttt{W, A, S, D} – pohyb vpřed, vlevo, vzad, vpravo
    \item \texttt{Shift} – běh
    \item \texttt{Mezerník} – skok
    \item \texttt{Myš} – otáčení pohledu
    \item \texttt{Kolečko} – změna FOV (30°–120°)
\end{itemize}

\section{Závěr}
Implementace naplňuje požadavky zadání a zároveň ukazuje silnou stránku ve fyzikálně korektním chování hráče, precizních kolizích a světelném modelu. Kód je rozdělen do samostatných modulů, což usnadňuje rozšiřování o další funkce (např. minimapa, AI, grafické efekty).

\end{document}
